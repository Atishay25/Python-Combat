To create or use your own template\+:


\begin{DoxyEnumerate}
\item Create a folder with the same name as your template (for example, {\ttfamily mycooltemplate}).
\item Within the template folder, create a file named {\ttfamily publish.\+js}. This file must be a Common\+JS module that exports a method named {\ttfamily publish}.
\end{DoxyEnumerate}

For example\+:


\begin{DoxyCode}{0}
\DoxyCodeLine{/** @module publish */}
\DoxyCodeLine{}
\DoxyCodeLine{/**}
\DoxyCodeLine{ * Generate documentation output.}
\DoxyCodeLine{ *}
\DoxyCodeLine{ * @param \{TAFFY\} data -\/ A TaffyDB collection representing}
\DoxyCodeLine{ *                       all the symbols documented in your code.}
\DoxyCodeLine{ * @param \{object\} opts -\/ An object with options information.}
\DoxyCodeLine{ */}
\DoxyCodeLine{exports.publish = function(data, opts) \{}
\DoxyCodeLine{    // do stuff here to generate your output files}
\DoxyCodeLine{\};}
\end{DoxyCode}


To invoke J\+S\+Doc 3 with your own template, use the {\ttfamily -\/t} command line option, and specify the path to your template folder\+:


\begin{DoxyCode}{0}
\DoxyCodeLine{./jsdoc mycode.js -\/t /path/to/mycooltemplate}
\end{DoxyCode}
 